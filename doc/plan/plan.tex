\documentclass[a4paper, twosided, 11pt]{scrartcl}

%% Language configuration
\usepackage{polyglossia}
\setdefaultlanguage[variant=swiss]{german}

%$ Specify where floatings go and make nicer looking tables
\usepackage{float}
\usepackage{array}
\usepackage{booktabs}
\usepackage{tabularx}

%% Set up font
\usepackage[T1]{fontenc}
\usepackage[usefilenames, DefaultFeatures={Ligatures=Common}]{plex-otf}
\renewcommand*\familydefault{\sfdefault}

%% Header and Footers
\usepackage[automark]{scrlayer-scrpage}
\ihead{Projectplan}
\chead{}
\ohead{\headmark}

\ifoot{}
\cfoot{}
\ofoot{\pagemark}

%% Tikz to draw stuff
\usepackage{tikz}
\usepackage{tikzsymbols}
\usetikzlibrary{automata, positioning, arrows, calc, shapes.geometric, fit}

%% Access metadata
\usepackage{titling}

% Metadata
\title{Projektplan}
\author{Naoki Pross, Joanne Robertz, Steven Wegmann, J\'er\^ome Roy}

% Document
\begin{document}
\begin{titlepage}
  \includegraphics[height=3cm]{pic/ost-logo}
  \begin{flushright}
    % TODO: OST logo
    \vspace{5cm}
    {\Huge \bfseries \thetitle} \\
    \vspace{5mm}
    {\LARGE Project: \textit{GlowingBroccoli}} \\
    \vspace{5mm}
    {\LARGE \bfseries Version 2.3}
  \end{flushright}
\end{titlepage}


\clearpage
\section*{\"Anderungsgeschichte}
\begin{tabularx}{\textwidth}{lrlX}
  \toprule
  \bfseries Datum & \bfseries Version & \bfseries Autor & \bfseries Beschreibung \\
  \midrule
  2020-10-05 & 1.0 & Wegmann S. & Dokument erstellt \\
  2020-11-02 & 2.0 & Pross N.   & Dokument bearbeitet \\
  2020-11-02 & 2.1 & Roy J.     & Dokument bearbeitet \\
  2020-11-02 & 2.2 & Robertz J. & Dokument bearbeitet \\
  2020-11-02 & 2.3 & Wegmann S. & Dokument bearbeitet \\
  \bottomrule
\end{tabularx}

\vfill
{
  \renewcommand{\arraystretch}{2}
  \begin{tabularx}{\textwidth}{lp{.2\textwidth}X}
                   & \bfseries Datum & \bfseries Unterschrift \\
    Pross Naoki    & \hrulefill & \hrulefill \\
    Robertz Joanne & \hrulefill & \hrulefill \\
    Roy J\'er\^ome & \hrulefill & \hrulefill \\
    Wegmann Steven & \hrulefill & \hrulefill \\
  \end{tabularx}
}

\clearpage
\tableofcontents
\listoffigures
\listoftables

\section{Team}

Das Team besteht aus folgenden Studenten:
\begin{itemize}
  \item Naoki Pross
  \item Joanne Robertz
  \item J\'er\^ome Roy
  \item Steven Wegmann
\end{itemize}

\section{Projektstrukturplan}

\begin{figure}[H]
  \centering
  \begin{tikzpicture}[
    node/.style = {
      rectangle, draw, very thick, fill=gray!20,
      inner sep = 10pt,
      % rounded corners = 1mm,
    },
    leaf/.style = {
      rectangle, draw = gray, very thick, fill=gray!20, anchor = west,
      minimum width = 4cm, minimum height = 1cm,
      rounded corners = 1mm,
    },
  ]

  \node[node] (prj) {\bfseries Glowing Broccoli};

  \node[node, below left = of prj.south west] (general) {Allgemein};
  \node[leaf, below right = of general.south west] (gen0) {Dokumentation};
  \node[leaf, below = of gen0] (gen1) {Systemkonzept};
  \node[leaf, below = of gen1] (gen2) {Systemtest};

  \foreach \i in {0, 1, 2} {
    \draw[very thick] ($(general.south)+(-.6,0)$) |- (gen\i.west);
  }

  \node[node, below right = of prj.south east] (sw) {Software};
  \node[leaf, below right = of sw.south west] (sw0) {Archikektur erstellen};
  \node[leaf, below = of sw0] (sw1) {Schlange zeichnen};
  \node[leaf, below = of sw1] (sw2) {Game Loop};
  \node[leaf, below = of sw2] (sw3) {Steuerung};
  \node[leaf, below = of sw3] (sw4) {Fr\"uchte};
  \node[leaf, below = of sw4] (sw5) {Kollision};
  \node[leaf, below = of sw5] (sw6) {Game Over};
  \node[leaf, below = of sw6] (sw7) {Menu \& Scoreboard};

  \foreach \i in {0, 1, 2, ..., 7} {
    \draw[very thick] ($(sw.south)+(-.6,0)$) |- (sw\i.west);
  }

  \draw[very thick] (prj.west) to[in=90, out=180] (general.north);
  \draw[very thick] (prj.east) to[in=90, out=0] (sw.north);

\end{tikzpicture}

\end{figure}

\section{Arbeitspakete}

\begin{table}[H]
\caption{Arbeitspakete}
\renewcommand{\arraystretch}{1.3}
\begin{tabularx}{\textwidth}{lX}
  \toprule
  \bfseries Allgemein & \bfseries Beschreibung \\
  \midrule
  Dokumentation & Dokumentation beinhaltet folgende Themen: Requirements
  Specification (separates Dokument), Beschreibung alle Software und Hardware
  APs und deren Resultat. Beinhaltet nicht: Benutzerhandbücher. \\

  Systemkonzept & Erstellen der Systemarchitektur mit deren Komponenten und Schnittstellen. \\

  Systemtest & Erstellen eines Systemtest welche alle User Anforderungen erfüllen. \\

  \midrule

  Architektur erstellen & Erstellen der Softwarearchitektur mit UML. \\

  Schlange mit Qt zeichnen & Die Schlange wird auf ein Window gezeichnet. \\

  Game Loop implementieren & Ein inkrementierender Timer aktualisiert die
  Position der Schlange. \\

  Steuerung hinzuf\"ugen & Der Spieler muss in der Lage sein, die Schlange mit
  der Tastatur zu steuern. \\

  Fr\"uchte & Auf der Karte wird zufällig ein Stück Nahrung generiert, das die
  Schlange fressen kann. \\

  Kollision der Objekte ermöglichen & Nach einer Kollision mit einer Frucht
  oder sich selbst wird ein Event ausgeführt. \\

  Game Over Status & Use Case Game Over gem\"ass SRS implementieren. \\

  Menu Status & Use Case Menu gem\"ass SRS implementieren. \\

  Spiel pausieren erm\"oglichen & Use Case Pause gem\"ass SRS implementieren. \\

  Scoreboard einf\"ugen & Use Case Scoreboard gem\"ass SRS implementieren. \\

  \bottomrule
\end{tabularx}
\end{table}

\section{Meilensteine}

\subsection{Debug / Normale}

\begin{enumerate}
  \item Grafische Oberfl\"ache und leere Game Loop realisieren
  \item Schlange bewegt sich und kann gesteuert werden
  \item Collision Checking und Game Over sind implementiert
  \item Score f\"ur ein Spiel wird gespeichert
  \item Menu und Scoreboard sind implementiert
  \item Spiel kann pausiert werden
\end{enumerate}

\subsection{Release}

\begin{enumerate}
  \item Spiel ist fertig und spielbar
  \item Framerate und Update-Rate sind stabil
\end{enumerate}

\section{Terminplan}
Siehe der Anhang \texttt{Gantt-Diagramm.pdf}.

\end{document}

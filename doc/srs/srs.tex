\documentclass[a4paper, twosided, 11pt]{scrartcl}

%% Language configuration
\usepackage{polyglossia}
\setdefaultlanguage[variant=swiss]{german}

%$ Specify where floatings go and make nicer looking tables
\usepackage{float}
\usepackage{array}
\usepackage{booktabs}
\usepackage{tabularx}

%% Set up font
\usepackage[T1]{fontenc}
\usepackage[usefilenames, DefaultFeatures={Ligatures=Common}]{plex-otf}
\renewcommand*\familydefault{\sfdefault}

%% Header and Footers
\usepackage[automark]{scrlayer-scrpage}
\ihead{}
\chead{}
\ohead{\headmark}

\ifoot{}
\cfoot{}
\ofoot{\pagemark}

%% Tikz to draw stuff
\usepackage{tikz}
\usepackage{tikzsymbols}
\usetikzlibrary{automata, positioning, arrows, calc, shapes.geometric, fit}

%% Access metadata
\usepackage{titling}

% Metadata
\title{Pflichtenheft}
\author{Naoki Pross, Joanne Robertz, Steven Wegmann, J\'er\^ome Roy}

% Document
\begin{document}
\begin{titlepage}
  \begin{flushright}
    % TODO: OST logo
    \vspace{5cm}
    {\Huge \bfseries \thetitle} \\
    \vspace{5mm}
    {\LARGE Project: \textit{GlowingBroccoli}} \\
    \vspace{5mm}
    {\LARGE \bfseries Version 3.1}
  \end{flushright}
\end{titlepage}


\clearpage
\section*{\"Anderungsgeschichte}
\begin{tabularx}{\textwidth}{lrlX}
  \toprule
  \bfseries Datum & \bfseries Version & \bfseries Autor & \bfseries Beschreibung \\
  \midrule
  2020-10-05 & 1.0 & Robertz J. & Dokument erstellt \\
  2020-10-19 & 2.0 & Robertz J. & Einleitung und allgemeine Beschreibung bearbeiten \\
  2020-10-19 & 2.1 & Wegmann S.  & Funktionale Anforderungen bearbeiten \\
  2020-10-19 & 2.2 & Roy J.     & Dokument bearbeiten \\
  2020-10-19 & 2.3 & Pross N.   & Dokument bearbeiten \\
  2020-10-22 & 3.0 & Pross N.   & Dokument in \textrm{\LaTeX} gesetzt \\
  2020-10-23 & 3.1 & Pross N.   & Use Cases bearbeiten \\
  \bottomrule
\end{tabularx}

\vfill
\section*{Unterschriften}
{
  \renewcommand{\arraystretch}{2}
  \begin{tabular}{lp{.4\textwidth}}
    Pross Naoki    & \hrulefill \\
    Robertz Joanne & \hrulefill \\
    Roy J\'er\^ome & \hrulefill \\
    Wegmann Steven & \hrulefill \\
  \end{tabular}
}

\clearpage
\tableofcontents
\listoffigures
\listoftables

\clearpage

\section{Einleitung}
\subsection{Zweck}

Im vorliegenden Dokument sind die Anforderungen definiert, welche im Projekt
``GlowingBroccoli'' umgesetzt werden müssen. Es beschreibt den Auftrag zwischen
Auftraggeber und Auftragnehmer. Der Ausdruck Pflichtenheft ist hier im Sinne
der IEEE Recommended Practice for Software Requirements Specification.
ANSI/IEEE Std 830-1998 verwendet. Die dort definierte Requirements
Specification beinhaltet sowohl die Benutzeranforderungen (Lastenheft gemäss
DIN 69901-5) als auch Realisierungsvorgaben an die Entwicklungsgruppe
(Pflichtenheft gemäss DIN 69901-5).

\subsection{Produk\"uberblick}
Unser Team hat sich dafür entschieden das klassische ``Snake'' Spiel zu
programmieren. Im Grunde geht es darum, durch bewegende Punkte, eine Schlange
aufzubauen, die immer grösser wird, umso mehr sie isst.
Als Abbruchbedingungen mit gefolgtem ``Game Over'' zählt:
\begin{itemize}
  \item Die Schlange berührt die Wand.
  \item Die Schlange frisst sich selber, zum Beispiel weil sie so lang geworden
    ist, dass es unmöglich ist sie im Spielfeld weiterhin agil zu bewegen.
\end{itemize}

\subsection{Definitionen, Akronyme und Abk\"urkungen}

\begin{description}
  \item[Widget] Diese Umgebung wird innerhalb des Fensters aufgebaut und dient
    um einen dynamischen Verlauf, wie z.B. ein Spiel, aufzubauen.
\end{description}

\section{Allgemeine Beschreibung}
\subsection{System\"ubersicht}
\begin{figure}[H]
  \centering
  \begin{tikzpicture}
  \draw[lightgray, step = 5mm, dotted] (0,0) grid (10, 5);

  %% Food
  \node at (8, 2) (food) {};
  \draw[thick, fill = red!50!black] (food) rectangle +(-.5, -.5);

  %% Snake
  \node at (3, 4) (snake-head) {};
  \draw[thick, fill = yellow!70!black] (snake-head) rectangle +(-.5, -.5);
  \foreach \i in {1, ..., 3} {
    \draw[thick, fill = green!70!blue]
      ($ (snake-head) + (-.5 * \i, 0) $)
        node (snake-body-\i) {}
        rectangle +(-.5, -.5);
  }

  \foreach \i in {0, ..., 4} {
    \draw[thick, fill = green!70!blue] ($ (snake-body-3) + (0, -.5 * \i) $) rectangle +(-.5, -.5);
  }

  %% Arrows
  \draw[<-, very thick] (food) to[controls = +(45:1) and +(-90:1)] +(1, 3.5)
    node[above] {\textbf{Food}};

  \draw[<-, very thick] (snake-body-2.north) to[controls = +(90:1) and +(-90:1)] +(2, 1.5)
    node[above] {\textbf{Snake}};

  \draw[->, ultra thick, green!50!black] ($ (snake-head.east) - (0, .25) $) to +(1, 0)
    node[right] {};
\end{tikzpicture}

  \caption[Beispiel f\"ur grafische Oberfl\"ache in Use Case Play]{Beispiel dafür, wie die grafische Oberfläche f\"ur das Use-Case Play aussehen sollte.}
\end{figure}

\subsection{Produktfunktionen} \label{sec:product-functions}
\begin{itemize}
  \item Das Spiel wird mittels einem Startknopf eröffnet.

  \item Die Schlange muss mittels den vier Pfeiltasten (oder \texttt{WASD})
    bewegt werden können.

  \item Das herumliegende Essen muss sich der Schlange anschliessen als Punkt,
    damit sie sich nach jedem Bissen auch vergrössert.

  \item ``Game Over'' muss erscheinen, wenn die Schlange aus einer bestimmten
    Abbruchsbedingung stirbt.

  \item Ein Scoreboard zeigt dem Benutzer die Ingame-Punkte der vorherigen Spiele an.
\end{itemize}

\subsection{Benutzereingenschaften}
Dieses Projekt soll Benutzergruppen ansprechen die gerne ihre Zeit in einen
lustigen Klassiker investieren möchten. Eine Altersbeschränkung gibt es hier
nicht.

\subsection{Einschr\"ankungen}
Keite.

\subsection{Annahmen und Abh\"angigkeiten}
Die Software soll auf allen der drei Hauptplattformen, nämlich Windows 10,
MacOS und Linux laufen. Dies kann durch \texttt{C++} als Programmiersprache und
das plattformübergreifende Qt Framework erreicht werden. Von der User Hardware
wird erwartet, dass sie die Fähigkeit hat einfache Grafiken darzustellen und
über eine moderne grafische Umgebung verfügt.

\subsection{Priorisierung der Anforderungen}
Die Funktionen, welche in \S\ref{sec:product-functions} ersichtlich sind,
beschreiben die Muss-Anforderungen welche benötigt werden, um den minimalen
Zweck der Software zu erfüllen. Als Soll-Anforderungen wird die Implementation
von optisch schöneren Grafiken beschrieben.  Die Wunsch-Anforderung beinhaltet
das Einfügen von weiteren Funktionen im Spiel, wie zum Beispiel verschiedene
Früchte oder die Implementation eines ``Scoreboards''.

\section{Externe Schnittstellen}
Keine, ausser Tastatur und Bildschirm.

\section{Funktionale Anforderungen}
\begin{figure}[h]
  \centering
  \begin{tikzpicture}[
    usecase/.style = {
      ellipse, draw, very thick, fill=gray!20,
      inner sep = 4pt, outer sep = 3pt
    },
    user/.pic = {
      \begin{scope}[line width = 1.5mm]
        \node[draw, circle, inner sep = .2cm] (head) {};
        %
        \draw (head.south) to +(0,-.2) node[inner sep = 0, outer sep = 0] (neck) {};
        \draw (neck.south) to +(0,-.5) node[inner sep = 0, outer sep = 0] (waist) {};
        %
        \draw (neck.center) to +(.5, -.2);
        \draw (neck.center) to +(-.5, -.2);
        %
        \draw (waist.center) to +(-.3, -.6);
        \draw (waist.center) to +(.3, -.6);
      \end{scope}
    },
  ]
  \begin{scope}
    \node[usecase] (menu) {Menu};
    \node[usecase, below = of menu] (game) {Play};
    \node[usecase, below = of game] (pause) {Pause};
    \node[usecase, right = of menu] (scoreboard) {Scoreboard};
    \node[usecase, right = of pause] (gameover) {Game Over};

    \begin{scope}[very thick]
      \draw[->, gray] (menu) to (game);
      \draw[<->, gray] (menu) to (scoreboard);
      \draw[<->, gray] (game) to (pause);
      \draw[<-, dashed] (game) to[out = 10, in = -120]
        node[pos = .7, above left] {\small extends} (scoreboard);
      \draw[<-, dashed] (game) to[out = -10, in = 120]
        node[pos = .7, below left] {\small extends} (gameover);
      \draw[->, gray] (gameover) to[out = 70, in = -70] (scoreboard);
    \end{scope}
  \end{scope}

  \node[very thick, draw, dashed, gray, fit = {
    ($(pause.south west) + (-1, -1)$)
    ($(scoreboard.north east) + (1, 1)$)
  }] (boundary) {};

  \node[above, gray] at (boundary.north) {\textbf{Snake Game}};

  \node[left = of boundary.west, outer sep = 5mm, xshift = -1cm] (player) {};
  \pic[at = {(player.center)}] {user};
  \node[below = of player] {Player};

  \draw[->, very thick] (player) to[controls = +(20:2) and +(180:2) ] (menu);
  \draw[->, very thick] (player) to[controls = +(0:2) and +(180:2) ] (game);
  \draw[->, very thick] (player) to[controls = +(-20:2) and +(180:2) ] (pause);

\end{tikzpicture}

  \caption{Use-case Diagramm}
\end{figure}

\subsection{Actors}
\begin{table}[H]
  \centering
  \begin{tabular}{ll}
    \toprule
    Actors & Beschreibung \\
    \midrule
    Player & Die Eingaben des Users beeinflussen den Verlauf des Spiels. \\
    \bottomrule
  \end{tabular}
  \caption{Liste der Actors}
\end{table}

\subsection{Kurzbeschreibung der Use Cases}
\begin{table}[H]
  \centering
  \begin{tabular}{lp{.75\linewidth}}
    \toprule
    Use Case & Beschreibung \\
    \midrule
    % \texttt{Exit} & Beim betätigen der Systemtaste \texttt{E} verlässt man das Spiel
    % sofort und gelangt auf die Hauptseite, wo sich wiederum der Startknopf
    % befindet. \\None, except keyboard and display.

    Menu & Der Spieler kann einen Namen eingeben und wählen, ob er spielen oder
    die \emph{Scoreboard} sehen möchte. \\

    Play & Das Spiel \emph{Snake} spielen. \\

    Pause & Beim betätigen der Systemtaste \texttt{P} oder \texttt{ESC} wird
    das Spiel pausiert. Beim erneuten drücken, geht das Spiel weiter. \\

    Game Over & Der Spieler hat das Spiel verloren, sein Punktestand wird auf
    dem Bildschirm angezeigt. \\

    Scoreboard & Es wird eine Liste der früheren Spiele desselben oder eines
    anderen Spielers angezeigt. \\

    \bottomrule
  \end{tabular}
  \caption{Kurzbeschreibung der Use Cases}
\end{table}

\subsection{Use Case Menu}
\paragraph{Vorbedingungen} Es gibt eine grafische Umgebung. Programm muss laufen.
% \paragraph{Nachbedingungen}
% \paragraph{Nicht-Funktionale Anforderungen}
\paragraph{Haputszenario} Spieler hat das Programm gestartet
% \paragraph{Unterszenarien}
\paragraph{Fehlerszenarien} Es gibt keine grafische Umgebung.
% \paragraph{Regeln}
% \paragraph{Anmerkungen}
% \paragraph{Beispiele}

\subsection{Use Case Play}
\paragraph{Vorbedingungen} Die Tastatur muss vorhanden sein.
% \paragraph{Nachbedingungen}
% \paragraph{Nicht-Funktionale Anforderungen}
\paragraph{Haputszenario} Die Spieler möchten ein Spiel spielen.
% \paragraph{Unterszenarien}
\paragraph{Fehlerszenarien} Es ist keine Tastatur vorhanden.
\paragraph{Regeln}
Die Schlange wird mit der Tastatur gesteuert. Die Schlange darf sich nicht aus
dem sichtbaren Bereich entfernen. Es ist der Schlange auch nicht erlaubt, ihren
eigenen Körper zu durchqueren. Wenn beides geschieht, kommt es zu einem Game
Over.

% \paragraph{Anmerkungen}
% \paragraph{Beispiele}

\subsection{Use Case Pause}
\paragraph{Vorbedingungen} Es muss ein laufendes Spiel geben.
% \paragraph{Nachbedingungen}
% \paragraph{Nicht-Funktionale Anforderungen}
\paragraph{Haputszenario}
Die Spieler wollen das Spiel unterbrechen, um etwas anderes zu tun. Und
vielleicht später wieder aufnehmen oder das Spiel beenden.
% \paragraph{Unterszenarien}
% \paragraph{Fehlerszenarien}
\paragraph{Regeln}
Wenn das Spiel andauert, wird es unterbrochen. Wenn das Spiel bereits
angehalten ist, sollte es wieder aufgenommen werden. Es sollte auch möglich
sein, das Spiel zu beenden und zum Hauptmenü zurückzukehren.
% \paragraph{Anmerkungen}
% \paragraph{Beispiele}

\subsection{Use Case Game Over}
\paragraph{Vorbedingungen} Es muss ein laufendes Spiel geben.
\paragraph{Nachbedingungen} Das Spiel endet.
% \paragraph{Nicht-Funktionale Anforderungen}
\paragraph{Haputszenario}
% \paragraph{Unterszenarien}
% \paragraph{Fehlerszenarien}
% \paragraph{Regeln}
% \paragraph{Anmerkungen}
% \paragraph{Beispiele}

\subsection{Use Case Scoreboard}
% \paragraph{Vorbedingungen}
% \paragraph{Nachbedingungen}
% \paragraph{Nicht-Funktionale Anforderungen}
% \paragraph{Haputszenario}
% \paragraph{Unterszenarien}
% \paragraph{Fehlerszenarien}
% \paragraph{Regeln}
% \paragraph{Anmerkungen}
% \paragraph{Beispiele}

\section{Sonstige Anforderungen}

\addcontentsline{toc}{section}{Referenzen}
\renewcommand{\refname}{Referenzen}
\begin{thebibliography}{2}
  \bibitem{ieee}
    IEEE Recommended Practice for Software Requirements Specification. ANSI/IEEE Std 830-1998
\end{thebibliography}

\end{document}
